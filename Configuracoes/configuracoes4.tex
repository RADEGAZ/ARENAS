 
% alterando o aspecto da cor azul
\definecolor{blue}{RGB}{41,5,195}
% informações do PDF
\makeatletter
\hypersetup{
	pagebackref=true,
	pdftitle={\@title}, 
	pdfauthor={\@author},
	pdfsubject={Modelo de artigo científico com abnTeX2},
	pdfcreator={LaTeX with abnTeX2},
	pdfkeywords={abnt}{latex}{abntex}{abntex2}{atigo científico}, 
	colorlinks=true,       		% false: boxed links; true: colored links
	linkcolor=blue,          	% color of internal links
	citecolor=blue,        		% color of links to bibliography
	filecolor=magenta,      		% color of file links
	backref=page,           % Para back ref
	urlcolor=blue,
	bookmarksdepth=4
}
\makeatother
%--
% O tamanho do parágrafo é dado por:
\setlength{\parindent}{1.25cm}
\usepackage{threeparttable} 
% Controle do espaçamento entre um parágrafo e outro:
\setlength{\parskip}{0.1cm}  % tente também \onelineskip

% Espaçamento simples
%\SingleSpacing

% Configurações do pacote backref
% Usado sem a opção hyperpageref de backref
\renewcommand{\backrefpagesname}{Citado na(s) página(s):~}
% Texto padrão antes do número das páginas
\renewcommand{\backref}{}
% Define os textos da citação
\renewcommand*{\backrefalt}[4]{
	\ifcase #1 %
	Nenhuma citação no texto.%
	\or
	Citado na página #2.%
	\else
	Citado #1 vezes nas páginas #2.%
	\fi}%

%--
 