%!TEX root = ..//Avali-Arena-Esportivas.tex
\section{ CONCLUSÃO }
\hspace*{1.25 cm} Este trabalho procurou demonstrar que o conhecimento técnico-científico do profissional (arquiteto, engenheiro) de avaliações pode ser aplicado na avaliação de bens e ativos incomuns, tal como o mercado de construção de arenas esportivas.\\ 
%
\hspace*{1.25 cm} O profissional de avaliações, além de estudar e ter conhecimento sobre o mercado em que está atuando, define o valor de mercado do bem de forma científica, através da formação de uma amostra representativa, do tratamento estatístico da mesma e da análise técnica criteriosa dos resultados obtidos, seguindo os procedimentos definidos pelas normas técnicas de avaliação publicadas pela ABNT (Associação Brasileira de Normas Técnicas).\\ 
%
\hspace*{1.25 cm}  Na referida amostra, foram considerados como corretos os valores de construção e/ou reforma dos estádios pesquisados. O mesmo procedimento foi adotado em relação às capacidades dos estádios, ou número de assentos.\\ 
%
\hspace*{1.25 cm}  Os estádios de futebol atuais são verdadeiras arenas multiuso, com tecnologia de ponta. Tais estádios podem ser considerados, inclusive, como espaços de consumo.