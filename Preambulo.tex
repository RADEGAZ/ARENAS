
% !TeX spellcheck = ptBR
% !TeX encoding = utf8
% !TeX program = lualatex
% -----------------------------------
\documentclass[
% -- op\c{c}ões da classe memoir --
article,			                                    % indica que é um artigo acadêmico
12pt,				                                   % tamanho da fonte
oneside,											% para impressão apenas no recto. Oposto a twoside
a4paper,											% tamanho do papel. 
% -- op\c{c}ões da classe abntex2 --
%chapter=TITLE,								  % títulos de capítulos convertidos em letras maiúsculas
section=TITLE,									 % títulos de se\c{c}ões convertidos em letras maiúsculas
%subsection=TITLE,							% títulos de subse\c{c}ões convertidos em letras maiúsculas
%subsubsection=TITLE % títulos de subsubse\c{c}ões convertidos em letras maiúsculas
% -- op\c{c}ões do pacote babel --
english,											% idioma adicional para hifeniza\c{c}ão
brazil,												% o último idioma é o principal do documento
sumario=tradicional
]{ltxdoc}

%\usepackage[top=120pt, bottom=2.5cm, left=3cm, right=2cm, headheight=95pt, footskip=66pt]{geometry}%
% PACOTES
%!TEX root = PREAMBULO.tex
\usepackage{lmodern}			% Usa a fonte Latin Modern			
\usepackage[T1]{fontenc}		% seleção de códigos de fonte.
\usepackage[utf8]{inputenc}		% determina a codificação utiizada (conversão automática dos acentos)
\usepackage{hyperref}  			% controla a formação do índice
\usepackage{parskip}			% espaçamento entre os parágrafos
\usepackage{microtype} 			% para melhorias de justificação
\usepackage{morefloats}			% permite mais floats
\usepackage{indentfirst} % identar o  primeiro paragrafo
\usepackage{lipsum}  % textos exparsos
% Babel e ajustes
\usepackage[brazil]{babel}		% idiomas
\addto\captionsbrazil{
	%% ajusta nomes padroes do babel
	\renewcommand{\bibname}{Refer\^encias}
	\renewcommand{\indexname}{\'Indice}
	\renewcommand{\listfigurename}{Lista de ilustra\c{c}\~{o}es}
	\renewcommand{\listtablename}{Lista de tabelas}
	%% ajusta nomes usados com a macro \autoref
	\renewcommand{\pageautorefname}{p\'agina}
	\renewcommand{\sectionautorefname}{se{\c c}\~ao}
	\renewcommand{\subsectionautorefname}{subse{\c c}\~ao}
	\renewcommand{\paragraphautorefname}{par\'agrafo}
	\renewcommand{\subsubsectionautorefname}{subse{\c c}\~ao}
	\renewcommand{\paragraphautorefname}{subse{\c c}\~ao}
}  

\usepackage{color}
 

% COMANDOS PROPRIOS
\newcommand{\abnTeX}{abn\TeX}
\newcommand{\abnTeXForum}{\url{http://groups.google.com/group/abntex2}}
\newcommand{\abnTeXSite}{\url{http://www.abntex.net.br/}}

\title{\textbf{A classe \textsf{abntex2}}: \\ \Large{Documentos
		técnicos e científicos brasileiros \\compatíveis com as normas ABNT}}

%   \thanks{Este documento
	%   se referete ao \textsf{abntex2} versão \fileversion,
	%   de \filedate.}

 


\EnableCrossrefs
\CodelineIndex
\RecordChanges

\changes{v1.0}{2013/02/01}{Versão inicial}
\changes{v1.9.3}{2015/01/26}{Release 1.9.3}
\changes{v1.9.4}{2018/06/06}{Release 1.9.4}

\usepackage[fontsize=12pt]{scrextend}
\usepackage{backref}
\usepackage{lastpage}     % para conseguir identificar a quantidade de páginas no preâmbulo
\usepackage{fancyhdr} 		    % para incluir headings e footers
% ------------------------------------------------------
\usepackage{fontspec} %% => Ligar fonte Arial
\setmainfont{Arial}
\usepackage{fontspec} %% => Ligar fonte Arial
\setmainfont{Arial}
% ------------------------------------------------------  													% Trazer os pacotes
\include{pacotes//pacotes2} 												% Trazer os pacotes
\usepackage{geometry}
\geometry{top=110pt, bottom=3.5cm, left=3.0cm, right=2.0cm, headheight=70pt}
\usepackage[utf8]{inputenc}
\usepackage{babel}
\usepackage[T1]{fontenc}
\usepackage{wrapfig,lipsum,booktabs}
\usepackage{graphicx}
\usepackage{fancyhdr}
\pagestyle{fancy}
\usepackage{tikz}
\usetikzlibrary{calc,positioning,arrows,shapes,shadows,fit,patterns,quotes,spy}
\usepackage{supertabular}  % Ppara tabelas em varias paginas
\usepackage{tabularx}
\usepackage{longtable}
\usepackage{pdflscape}
\usepackage[fontsize=12pt]{scrextend}
%%%%%%%%%%%%%
\usepackage{multirow}
\usepackage{setspace}                                                        % Para espacamentos
\usepackage{varioref}                                                           % referencia cruzada
\usepackage{enumitem}  													% para melhoria nos itens numerados
\usepackage{xcolor} 															% Para cores personalizadas
\usepackage{wrapfig}
\usepackage{float} 								% Força o posicionamento da figura
\usepackage[]{backref}	                         %  VOLTAR com as citações
\usepackage{hypernat}
\usepackage{etoolbox}
%\usepackage{threeparttable} 
\usepackage[flushleft]{threeparttablex}
\usepackage{booktabs}
\usepackage{multirow}
 												% Trazer os pacotes
%%%CONFIGURACOES
\setlength{\parindent}{0cm} % SEM IDENTECAO

\include{Configuracoes//configuracoes4} 
\setlength{\parindent}{0cm} % SEM IDENTECAO


%\renewcommand{\footrulewidth}{1pt}

%\geometry{a4paper, includehead, top=-0.4cm, left=2cm}
\setlength\headwidth{\paperwidth}
\fancyhead{} % Limpar all fields cabecalho
\fancyfoot{} % Limpar all fields rodape
\pagestyle{fancyplain}                    % aplicando o estilo 
\fancyhfoffset[L]{0.2cm}\fancyhfoffset[R]{0.2cm}

\chead{\includegraphics[width=\headwidth]{pre-pos-textuais/cabecalho}}
\renewcommand{\headrulewidth}{0pt}

%\clearpage\begingroup\pagestyle{empty}\cleardoublepage\endgroup

\usepackage{lastpage} % number of last page 
\rfoot{Page \thepage\ de \pageref{LastPage}}
\cfoot{\includegraphics[width=0.8\headwidth]{pre-pos-textuais//rodape}}

\renewcommand{\footrulewidth}{0pt} 
\renewcommand{\headrulewidth}{0pt}
\usepackage{lastpage} % number of last page 
\rfoot{ \thepage\ de \pageref{LastPage}}


\makeatletter
\patchcmd{\BR@backref}{\newblock}{\newblock(~}{}{}
\patchcmd{\BR@backref}{\par}{)\par}{}{}
\makeatother

\renewcommand{\backrefxxx}[3]{(page \hyperlink{page.#1}{#1})} 
% ------------------------------------------------------
\usepackage{fontspec} %% => Ligar fonte Arial
\setmainfont{Arial}
\usepackage{fontspec} %% => Ligar fonte Arial
\setmainfont{Arial}
% ------------------------------------------------------ 

\begin{document}

\begin{center}
	\textbf{Dinâmica espaço-temporal da contaminação pelo rompimento da Barragem de Fundão próximo à foz do Rio Doce}
\end{center}
	\thispagestyle{empty} % Não enumere esta pagina-
	
%

%\listoftables

%!TEX root = ..//Avali-Arena-Esportivas.tex
 
\begin{center}
	\textbf{XXIII COBREAP - CONGRESSO BRASILEIRO DE ENGENHARIA DE AVALIAÇÕES E PERÍCIAS - JOÃO PESSOA - 2025}
\end{center}
\begin{center}
	\textbf{TRABALHO DE AVALIAÇÃO}
\end{center}

Avaliação de arena esportiva pelo método comparativo direto de dados de mercado, com a aplicação do tratamento científico\\ 
RESUMO\\ 
Arenas esportivas são locais multifuncionais projetados para sediar uma variedade de eventos, incluindo competições esportivas, shows, conferências, futebol e outros tipos de entretenimento.\\ 


Este trabalho pretende de forma objetiva e tão simples quanto possível, estabelecer uma marcha numérica baseada através da Inferência estatística, pelo Método Direto Comparativo de Dados de Mercado - MDCDM, com o conhecimento técnico da engenharia de avaliações, mostrando que estes espaços podem ser avaliados por ele.\\ 
Palavras-chave: Ativos singulares, Arenas poliesportivas, Futebol

% ------
 \newpage
 \tableofcontents
 
\newpage
% ------
%!TEX root = ..//preambulo.tex

\section{Introdução}

\hspace*{1.25 cm} O rompimento da barragem de Fundão, no povoado de Bento Rodrigues, município de Mariana, Minas Gerais, em 5 de novembro de 2015, lançou rejeitos de mineração de ferro na bacia hidrográfica do Rio Doce, causando uma série de prejuízos, incluindo perdas de vidas humanas e contaminação ambiental a jusante do corpo hídrico. \\
%
\hspace*{1.25 cm} O volume de rejeitos liberado, contendo material estranho ao bioma característico da região, alterou simultaneamente as condições dos corpos hídricos e da vegetação marginal..\\
%
\hspace*{1.25 cm}A cobertura desse evento foi amplamente noticiada pela imprensa nacional e internacional, incluindo veículos como \textit{The Wall Street Jornal}, \textit{The Guardian}, \textit{Le Monde}, entre outros.\\
%
\hspace*{1.25 cm} Em acordo com o Ministério publico (MP) e sua controladora a anglo-australiana  \textit{BHP Billinton} ,  baseado nos princípios de poluidor pagador, trazidos no artigo 3º, inciso IV, da Lei 6.938/81 (Politica Nacional do Meio Ambiente), estabeleceram um termo de pactuação e posterior outra repactuação.  A repactuação atendendo o principio do poluidor-pagador, visa o dever de corrigir, recuperar e/ou eliminar os efeitos negativos ja produzidos em uma contaminação antrópica. \\
% 
\hspace*{1.25 cm}  A repactuação, utilizando ferramentas de cartografia , acordo com \cite[p~79]{Magri}, determinando ações de reparação aos atingidos e a  amplitude de localização geográfica em que ocorram ações de contaminação, e  por meio de direta interveniência  judicial e administrativa, e a fundação criada para reparação do rio Doce, denominada Fundação Renova.\\
 %
\hspace*{1.25 cm} Passados nove anos, com base na data de 23 de maio de 2025, e com a redução da atenção midiática sobre o desastre, este estudo propõe como objeto de pesquisa avaliar se os efeitos da contaminação por rejeitos ainda podem ser detectados por sensores orbitais e se há alterações passíveis de observação por esses instrumentos. Esses aspectos são objetos de perícia ambiental, conforme descrito por Arantes (p. 130). \cite[p.130]{Arantes} \\
 %
\hspace*{1.25 cm}  A justificativa deste estudo fundamenta-se na delimitação estabelecida pela repactuação entre o Ministério Público e a Fundação Renova. Além disso, considera-se a análise dos enquadramentos dos corpos d’água e as diretrizes ambientais previstas na Resolução Conama nº 357, de março de 2005, alterada pelas Resoluções nº 410/2009 e nº 430/2011, que podem ser observadas em locais dentro e fora do âmbito da repactuação.\\
%
\hspace*{1.25 cm} A tragédia do rompimento da barragem de Fundão, em Mariana/MG, ocorrida em 5 de novembro de 2015, representa um dos maiores desastres socioambientais do Brasil. Este evento catastrófico lançou aproximadamente 39 milhões de metros cúbicos de rejeitos de mineração na Bacia do Rio Doce, causando a perda de 19 vidas e impactando severamente populações em dezenas de municípios até a foz no Espírito Santo. A amplitude do desastre e seus prejuízos humanos e ambientais ganharam notoriedade na imprensa nacional e internacional, como The Wall Street Journal, The Guardian e Le Monde.\\
%
\hspace*{1.25 cm} Em resposta à calamidade, as partes envolvidas – a Samarco, suas acionistas Vale e BHP Billiton, a União e os governos de Minas Gerais e do Espírito Santo – firmaram o Termo de Transação e Ajustamento de Conduta (TTAC) em 2 de março de 2016. Com base neste acordo, foi criada a Fundação Renova em 2016, uma entidade privada com a missão de conduzir as 42 ações e programas socioeconômicos e socioambientais definidos para a reparação. No entanto, ao longo dos anos, a atuação da Fundação Renova foi alvo de diversas críticas e gerou um passivo significativo de 85 mil processos judiciais, evidenciando a necessidade de uma solução mais eficaz e abrangente.\\
%
\hspace*{1.25 cm} Diante da complexidade e da insatisfação com o progresso reparatório, iniciaram-se em março de 2021 as tratativas para uma renegociação ampla dos acordos, formalizadas pela Carta de Premissas em 22 de junho de 2021. Este processo, conduzido inicialmente pelo Conselho Nacional de Justiça (CNJ) e, a partir de agosto de 2022, sob a liderança do Tribunal Regional Federal da 6ª Região (TRF6) através da Mesa de Repactuação, buscou encerrar os múltiplos litígios por meio de um procedimento de conciliação. Após quase três anos de intensas negociações, o Acordo Judicial para Reparação Integral e Definitiva Relativa ao Rompimento da Barragem de Fundão foi finalmente assinado em Brasília em 25 de outubro de 2024 e homologado por unanimidade pelo Supremo Tribunal Federal (STF) em 6 de novembro de 2024.\\
%
\hspace*{1.25 cm} Este novo acordo, que substitui integralmente o TTAC de 2016 e seus aditivos, busca a reparação integral e definitiva de todos os danos socioambientais e socioeconômicos. Um dos seus pilares é a extinção da Fundação Renova e do Comitê Interfederativo (CIF), transferindo a responsabilidade integral pelas ações de reparação diretamente para a Samarco, que iniciará um período de liquidação para essa transição. O acordo prevê um valor econômico total de R\$ 170 bilhões, que inclui:
\begin{description} [itemsep=1pt,parsep=1pt]\vspace{0.00mm} 
	\item[•] R\$ 38 bilhões já desembolsados desde a tragédia.
	\item[•] R\$ 100 bilhões em "dinheiro novo", destinados aos entes públicos para custeio de medidas compensatórias e projetos socioambientais e socioeconômicos.
	\item[•] R\$ 32 bilhões em "Obrigações de Fazer", que incluem indenizações individuais, reconstrução de comunidades e recuperação de áreas degradadas, sem um teto financeiro pré-determinado, devendo a Samarco comprovar a conclusão de cada obrigação.
\end{description}

\hspace*{1.25 cm}
A nova governança do processo reparatório será marcada pela transparência, com a criação de um "Portal Único" denominado "Reparação Rio Doce", onde todas as partes envolvidas (signatários) serão responsáveis pela atualização dos dados, permitindo à sociedade civil acompanhar detalhadamente a implementação do acordo. A homologação do acordo também resultará na extinção de inúmeras ações judiciais e procedimentos administrativos, visando a definitiva resolução dos litígios.

\hspace*{1.25 cm} Com base na exposição acima, as seguintes hipóteses são formuladas:
\begin{description} [itemsep=1pt,parsep=1pt]\vspace{0.00mm} 
	\item[$H_{1}$:\label{h1}] O sensor orbita Landsat 8 consegue identificar alterações em respostas em radiância na época  do ocorrido, e seus efeitos ainda podem ser mensurados
	\item[$H_{2}$:\label{h2}] Se este sensor orbital consegue distinguir alterações espectral  do fenômeno, o mesmo consegue determinar sua amplitude espacial geográfica.
	\item[$H_{3}$:\label{h3}] Existe resquícios desta contaminação no local, e que podem ser mensurados.
\end{description}

% ------
%!TEX root =..//preambulo.tex

\section{ Objetivos}
%
\subsection{ Objetivos Gerais}
%
\hspace*{1.25 cm} O objetivo deste estudo é comparar medidas de reflectância no solo, em vegetações e na água, utilizando a tecnologia de sensoriamento remoto (SR), para diferenciar as condições geográficas antes, durante e após o evento em análise. Além disso, busca-se verificar a possibilidade de identificar locais com contaminações e/ou alterações antrópicas decorrentes do desastre de Mariana
%\hspace*{1.25 cm} Com , verificar se é possível discriminar locais com contaminações resultantes do desastre de mariana. 
 % 
\subsection{ Objetivos Específicos}

 \hspace*{1.25 cm} A estruturação das informações será dividida em etapas. Inicialmente, serão obtidos dados por meio de imagens orbitais, armazenados, processados e analisados.Para isso, será utilizada a plataforma de acesso livre da "\textit{Google Earth Enginer(GEE)}". E como descreve em \cite[p.1]{Mutanga}, motivados por essa plataforma permitir o acesso a séries de imagens Landsat a partir de 2008.\\
%
 \hspace*{1.25 cm} O segundo passo consiste na realização de uma amostragem sistemática em locais geográficos específicos, utilizando valores de reflectância obtidos pela plataforma GEE. Essa abordagem possibilitará a construção de um painel temporal da ocorrência do evento.\\
% 
 \hspace*{1.25 cm} Por fim, a modelagem do fenômeno será realizada com base em critérios de validação estatística, que fornecerão subsídios para uma amostragem mais refinada, utilizando métodos probabilísticos aplicados a classes de uso do solo em zonas específicas\\
% 
% \hspace*{1.25 cm} Por fim, a obtenção de amostras in-loco de informações sobre padrões de indicadores químicos
 
 % \hspace*{1.25 cm} \textcolor{gray!22}{\lipsum[1-2] }\cite{IJSN}
  

	

% ------
%!TEX root = ..//preambulo.tex

\section{Material e Método}

 \subsection{Local de Estudo}

  \hspace*{1.25 cm}  A região estudada está localizada na microrregião administrativa central do estado do Espírito Santo, próxima ao delta do rio Doce, em uma planície fluvial composta por depósitos quaternários, dividida em porções de natureza lagunar e fluvial, ambas caracterizadas por processos de acumulação. Os solos predominantes na área incluem Cambissolos Eutróficos e Organossolos, sem influência marinha.\\
  %
     \begin{wrapfigure}{r}{0.50\textwidth}
	\begin{center}
		\centering \small \caption{Zona hachurada, região da repactuação}
		\includegraphics[width=0.96\linewidth]{FIGURAS/Localizacao}
		\label{fig:localizacado}\\{Fonte: Elaborado pelos Autores (2025)}
	\end{center}
\end{wrapfigure}
  \hspace*{1.25 cm}  O relevo da região é predominantemente plano, com declividade inferior a 2\% em toda a sua extensão. Quanto ao uso do solo, podem ser identificadas seis classes principais: (1) áreas de cultivo com culturas como mamão, coco-da-baía, cana-de-açúcar e abacaxi, entre outras; (2) áreas de reflorestamento, em proporção semelhante às áreas cultivadas; (3) pastagens; (4) brejos; (5) porções esparsas de solo exposto; e (6) pequenas áreas de corpos d’água, como açudes. \\
  %
%
  %
  \hspace*{1.25 cm} A escolha desta região como objeto de estudo justifica-se pela diversidade de usos do solo mencionada, que permite a análise da área sob a perspectiva da repactuação ambiental. Além disso, a região possibilita a observação de eventos climáticos extremos, como inundações (conforme ilustrado na Figura \ref{fig:indao}), e o estudo do encontro do fluxo de rejeitos com a foz do rio Doce e o oceano Atlântico, destacado nas figuras correspondentes. \ref{fig:indao2} e \ref{fig:indao3}\\  
% {{{   
			\begin{minipage}[t!]{0.33\textwidth}
				\begin{figure}[H]
					\centering \small \caption{Ano de 2013}
					\includegraphics[width=0.97\linewidth]{FIGURAS/enche}
					\label{fig:indao} 
				\end{figure}			
			\end{minipage}\hfill
			\begin{minipage}[t!]{0.33\textwidth}
				\begin{figure}[H]
					\centering \small \caption{Ano de 2016}
					\includegraphics[width=0.97\linewidth]{FIGURAS/lamache1}
					\label{fig:indao2} 
				\end{figure}					
			\end{minipage} 
			\begin{minipage}[t!]{0.33\textwidth}
				\begin{figure}[H]
					\centering \small \caption{2016 NIR}
					\includegraphics[width=0.97\linewidth]{FIGURAS/lama1red}
					\label{fig:indao3} 
				\end{figure}					
			\end{minipage} 
			\begin{center}
				Fonte:   Elaborado pelos Autores (2025)
			\end{center}
			% }}}

  \hspace*{1.25 cm} Com base na análise da situação física e considerando que o objeto da repactuação abrange uma área de 5.723,26 km², o local de estudo aprofundado corresponde a uma extensão de 10.000 hectares. Assim, o instrumento mais adequado para a pesquisa é o uso de imagens aéreas ou orbitais, com preferência por estas últimas. O delineamento da pesquisa será amostral, de caráter qualitativo-descritivo, com a variável mensurada longitudinalmente em um painel temporal\\
  %
    \hspace*{1.25 cm} Para minimizar questionamentos técnicos ou sociais, a metodologia deve ser delineada de forma totalmente replicável, tanto em sua estrutura quanto em seu conteúdo. A base teórica será fundamentada, majoritariamente, em publicações internacionais, como livros e artigos científicos. As imagens utilizadas como base de dados serão disponibilizadas gratuitamente à comunidade científica. Além disso, os aspectos substantivos serão priorizados, com afirmações fundamentadas em resultados matemáticos. Por fim, a pesquisa adotará uma abordagem adjetiva, classificando a situação antes, durante e após o evento extrínseco.

 \subsection{Sensoriamento Remoto}

  \hspace*{1.25 cm} A obtenção de dados obtidos por sensoriamento remoto, o principio básico de sensoriamento remoto, disposto por \cite[p.~24]{Reddy} é "ciência e a arte de se obter informação sobre um objeto, área ou  fenômeno o através de uma analise de dados adquiridos sem o contado com o objeto, fenômeno que se está sendo investigado".  Acrescenta-se ainda que em \cite[p.~7]{Jensensens} descreve que além das coordenadas x,y,z e profundidade, o sensoriamento remoto possibilita a analise da biomassa, a temperatura, e suas misturas contidas, e ainda outras. \\
  %
  \hspace*{1.25 cm} De mesma forma em \cite[p.1]{Lilesat}, "\textit{a ciência e arte de obter informações a respeito de um objeto, área ou fenômeno pela análise de dados adquiridos por um sistema que não se encontra em contado com o objeto , área ou fenômeno sob investigação}" 
  
 \hspace*{1.25 cm} A captação dessas informações, ocorre por meio de dados, reflectividade do objeto observado - superfície terrestre -, ocasionado pela emissão de energia eletromagnética emitida pelo sol, sendo parte absorvida pela  superfície terrestre e parte reflectida a atmosfera. e em ambos efeitos são  captados  por sensores orbitais.\\
 %
  \hspace*{1.25 cm} Diferentes sensores, com atributos e processos construtivos diferentes e  inerentes, segundo \cite[p.~20]{Centeno} armazenarão o espectro eletromagnéticos, comprimento de ondas de variável contínua, em canais e/ou bandas especificas. Esses canais, comprimento de ondas em faixa distintas divididas e  classificadas, a partir deste momento em variável discreta,  são denominadas \textbf{\textit{resolução espectral}}, ainda em \cite[p.~54]{Centeno} a maior ou menor capacidade de registrar diferenças espectrais entre os alvos e a medida desejada. Considerando as perdas atmosféricas, variações temporais, e as propriedades inerentes dos corpos imageados; composições químicas. \\
%
  %
  \hspace*{1.25 cm} O processo  de manipulação e classificação da variável discreta e seu espaço multidimensional e sua coordenada  \textbf{DN(x,y)}, como em \cite[p.~183]{Schowengerdt}, possui nome próprio denominado em linguá inglesa  "\textbf{\textit{Spectral Tranforms}}". E dentro destas operações matemáticas para se obter melhorias da informação destas imagens, agora em \cite[p.~485]{Lilesat}, sejam estas operações: manipulação de contrastes, correções geométricas, correções radiométricas , classificação entre outras. Existe um processo a ser desenvolvido neste estudo, denominada diferenças normalizadas e índices.\\
%
%
 \hspace*{1.25 cm} Voltando a \cite{Centeno} e este citando \cite[p.339]{Chuvieco}, é a combinação entre bandas, obtendo uma nova imagem, com características distinta, de forma a realçar a informação dos alvos e suas variações químicas e bióticas. \\
%
\hspace*{1.25 cm} Esta imagem, georreferenciada, mesmo sem o processo de classificação automática ou manual, pode ser entendida em estar em  perspectiva tri-dimensional (3d) por banda com características (x,y, valor discreto), mas pode incorporar  a quarta dimensão (4d) o tempo.
\begin{figure}[H]
	\centering  \small \caption{Quatro dimensões  em cubo de dados : x,y,banda, e tempo}
	\includegraphics[width=0.97\linewidth]{FIGURAS//quatroDimensao}
	%\caption{\href{file:./DIAGRAMAS/flow-diagrama-Alterado2.tex}{TEX File} }
	\label{fig:quatroDimensao}{ Fonte:  Em \cite[p.60]{Pebesma} }
\end{figure}
  \hspace*{1.25 cm} A partir do conceito, que compreendemos como cubo de dados, deve-se ter em mente que a variação da atributo do objeto a ser mensurado, este deve ser coletado. Então  surge a necessidade de amostragem, melhor dizendo, um plano de amostragem, livre de vieses, este detectando a variedade temporal, em sua estrutura de posicionamento, incluindo a principal variedade medida, que em nosso caso é a reflectância e suas composições e combinações .
%
 \subsection{Amostragem}
 %
%
  \hspace*{1.25 cm} O conceito de amostragem, e mais especificamente uma amostra, para  \cite[p.~281]{Krishnaswamy} deve ser utilizado quando \textbf{\underline{não é possível}} \textbf{ou prático} utilizar observação do fenômeno estudado em toda a população, este empecilho pode ser ocasionado por diferentes formas, sejam ele associados a localização geográfica, população numerosa, ou mesmo custo de aquisição elevado.\\
 %
 %
  \hspace*{1.25 cm}  Acrescenta-se a estes fenômenos, temos o tempo, pois se a alteração já ocorreu existe a dificuldade de rever eventos que foram alterados,e por isso devemos ter métodos de obtenção desta informação na época estuda, e por isso utilizamos ás técnicas de sensoriamento remoto, e processamento de imagem  para obtermos a informação.\\
  %
    %
  \hspace*{1.25 cm} E ao estabelecimento do plano de amostragem, devemos ter em mente o conceitos de esta atividade o  \cite[p.2]{Bolfarine1},  o pesquisador planeja, executa, corrige e analisa o procedimento a ser proposto e usado. Tendo como base a técnica de mensuração, e a questão respondida.\\
% 
 %
  \hspace*{1.25 cm} A escolha do método de amostragem livre de viés, e que o erro de estimativa seja minimizado, está diretamente ligado a estrutura de amostragem.  Em nosso estudo optamos \underline{primeiramente} pela estratificada aleatória por classe de uso do solo, de acordo com \cite[p.191]{Ariza}, e  \underline{posterior}  de  sistemática, estabelecimento de um \textbf{grid}, por grades regulares.\\
  %
   %
  \hspace*{1.25 cm}  O objetivo da primeira técnica, segundo  \cite[p.22]{Lohr} quando o tipo de amostragem por estratificação pode ser utilizada em regiões especificas que desejamos obter informação, em nosso caso a variação de reflectância temporal em classes de uso do solo especifica. Já na segunda  possibilitando que todos os fenômenos mensurados, tenham a mesma possibilidade de ser coletado, e em ambas livres de vieses.\\
  %
                \begin{wrapfigure}{l}{0.65\textwidth}
  	\begin{center}
  		\centering  \small \caption{Amostragem em classes}
  		\includegraphics[width=0.97\linewidth]{FIGURAS/usoSOLOamostras}
  		\label{fig:usoSOLOamostras}\\{ Fonte:   Elaborado pelos Autores (2025)}
  	\end{center}
  \end{wrapfigure} 
  %
 \hspace*{1.25 cm} A metodologia de planejamento e direcionamento baseada na relação entre amostras e classes de uso do solo é um procedimento comum entre profissionais da área de cartografia. Diversos estudos descrevem essas atividades; no entanto, segundo os autores, os trabalhos mais relevantes destacam os conhecimentos apresentados por Congalton \cite[p.79]{Congalton}, que vinculam cada classe de uso do solo à quantidade de amostras correspondente. Além disso, Ariza \cite[p.135]{Ariza} complementa ao determinar o tamanho amostral e o erro com base na estimativa, enquanto \cite[p.192-196]{Ariza} ilustra os quantitativos, em metros quadrados, de cada classe de uso do solo e o número necessário de fragmentos por classe. Esses fundamentos permitem a aplicação de fórmulas estatísticas, como a equação \( n= \dfrac{zC ^{2} * sd^{2}* N}{ E^{2}(N-1) + Zc^{2}*sd^{2}} \) para determinar o número de amostras por classe, considerando o erro, o nível de confiança estipulado e a representação gráfica demonstrada na Figura \ref{fig:usoSOLOamostras}  .\\
 %
   %  \hspace*{1.25 cm}  Segundocorroborando a escolha do plano de amostragem estratificada, em outros termos: segmentação de grupos em estratos(ou subpopulações),  é o conhecimento prévio da área de estudo, possibilitando que as amostras sejam suficiente e se destaque eventos mais importantes.\\
    

  \subsection{Geoestatística} 

 \hspace*{1.25 cm} Segundo \cite[p.1]{delgado}  a modelagem geoestátistica, tem um papel chave na análise da evolução de eventos distribuídos sobre a superfície da terra e ciências da engenharia ligadas a recursos naturais. \\
 %
   \hspace*{1.25 cm} A  utilização de malha regularas, também conhecido como "\textbf{grid}" , como exposto em \cite[p.83]{Andriotti}, tem a função de através de pontos amostrados, estabelecer um mapa de forma continua, passível de estimar valores dentro deste estrutura e até estabelecer extrapolações. Isso é possível através de técnicas de interpolação, voltando a \cite[p.304]{Ariza},  caracterizando o papel da interpolação, para este autor, "\textit{a interpolação e uma transformação que permite estivar valores desconhecidos a partir de valores conhecidos em posições certas}". \\
  %
    \hspace*{1.25 cm} Isso ocorre por que, conforme \cite[p.21]{Yamamoto}, o fenômeno espacial compreende um conjunto de valores possíveis da variável de interesse, e a sua distribuição e variabilidade está dentro de respectivos domínios sejam eles 2D ou 3D.  Voltando a  \cite[p.95]{Andriotti}, pode ser caracterizada como variável regionalizada.\\
  %
  \hspace*{1.25 cm}  Em  \cite[p.37]{Webster} os fenômenos físicos naturais, possuem variação determinística e sistemática, e estão relacionados a campo da matemática e geometria. E através destas das técnicas, podemos estabelecer predicações e interpolações entre elas a polígonos de Thielsse, inverso da distancias, superfícies de tendencias, splines e por fim a krigagem.   
  %
  \subsection{Dados Espaço - Temporal} 
  
  %
  \hspace*{1.25 cm} O paradigma espaço tempo, para \cite[p.2]{mateu}, este processo pode ser assumido como um modelo estatístico, e ser representado pela equação:
  \begin{equation}
  	 Z(x,s,t) =  \eta ( x,(s,t), s.t.\beta) + \in (x,s,t)  \; s \in D, t \, \in T.  
  \end{equation}
  %
  \hspace*{1.25 cm}   Onde $s$ corresponde a uma localização espacial, $t$ um momento temporal, $x$ potencialidades  dependentes de regressões , e $ \eta $ uma parametrização do modelo.\\
  %
   \hspace*{1.25 cm} Para outro autor,  \cite[p.151]{Bivand} o interesse e local estudado, pode ser separado em: ponto ou área, tempo ou intervalo, e por fim a ocorrência medida. Já que possuem diferentes  propriedade e qualidade também distintas  \\
   %
  \hspace*{1.25 cm} A etapa inicial consiste na obtenção de dados (imagens), que podem ser adquiridos por meio de plataformas como o Google Earth Engine (GEE), agilizando o processo de produção de informações. Esses dados podem ser integrados à implementação de scripts que automatizam a busca por cenas em datas específicas, permitindo também o cruzamento com outras informações para análises mais precisas, como a avaliação da cobertura e uso do solo atual (Figura \ref{fig:usoSOLOamostras}). Além disso, é possível selecionar aspectos como o percentual de cobertura de nuvens e o intervalo espaço-temporal das informações obtidas.
  %
  %
  \lstset{
  	language=Java, % Define a linguagem como JavaScript
  	caption=Código de obtenção de imagens multiespectrais Landsat8 plataforma Google Earth Engine Code\, em linguagem JavaScript.,} % Legenda do código
  
  \begin{lstlisting}
  	var dataset = ee.ImageCollection('LANDSAT/LC08/C02/T1_L2')
  	.filterDate('2013-05-01', '2025-05-01');
  	// Applies scaling factors.
  	function applyScaleFactors(image) {
  		var opticalBands = image.select('SR_B.').multiply(0.0000275).add(-0.2);
  		var thermalBands = image.select('ST_B.*').multiply(0.00341802).add(149.0);
  		return image.addBands(opticalBands, null, true)
  		.addBands(thermalBands, null, true);   	}
  	dataset = dataset.map(applyScaleFactors);
  	var visualization = {
  		bands: ['SR_B4', 'SR_B3', 'SR_B2'],
  		min: 0.0,
  		max: 0.3,  	};
  	Map.setCenter(-39.93696, -19.5597, 8);
  	Map.addLayer(dataset, visualization, 'True Color (432)');
  \end{lstlisting}
 %=========================================================== 
  \subsection{Operações}  
  %
    \subsubsection{Indices Espectrais}
 \hspace*{1.25 cm} A operação com dados raster, dados binários, para \cite[106]{Dormam} "\textit{grids de valores numéricos}", e estes estão agrupados bandas espectrais, e para \cite[p.178 e p.181]{Liu}  podem sofrer operações e conversões unárias, e sujeitas a operações comuns matemáticas com: adição, subtração, multiplicação, divisão, módulos. Inclui-se também operações relacionais e boleanas, logicas e combinatórias.  \\
 %
 \hspace*{1.25 cm}  Neste contexto, inicia-se nossas atividades. com a compreensão de índices espectrais, devemos entender, que são operações matemáticas entre bandas, sendo amplamente difundidos em diferentes meios das literatura técnica. Ao nosso estudo de alteração do meio físico, e conhecendo que os principais elementos encontrados no local são: solo, a água e a vegetação, entende-se  que  os índices  são capazes meios capazes de ser sensibilizados pelas variações destes elementos. E ao utilizamos as operações algébricas  \eqref{eq:ndvi}, \eqref{eq:ndwi} e \eqref{eq:savi} , de combinação de bandas que estão dispostas em \cite[p.165]{Thekapbail} e equação  \eqref{eq:evii} \cite[p.7]{Thekapbail}. 
 %
 \begin{equation} \label{eq:ndvi}
 	 NDVI = \dfrac{nir - red}{ (nir + red) }
 \end{equation}
 \begin{equation}\label{eq:ndwi}
	NDWI = \dfrac{green - nir}{ (green + nir) }
\end{equation}
 \begin{equation}\label{eq:evii}
	EVI = \dfrac{G*(nir - red)}{ nir + 2,4. red +1) }
\end{equation}
 \begin{equation}\label{eq:savi}
	SAVI = \dfrac{(nir - red)*(1 +l)}{ (nir + red +L) }
\end{equation}
%
\hspace*{1.25 cm} Onde, "red" corresponde a banda vermelha, "green" a banda verde, "NIR" o infra vermelho proximo, e o "L" correção de reflectâncias do solo.\\
%
 \hspace*{1.25 cm} Para manipulação destes dados utilizando a linguagem R na versão 4.5, e ambiente de desenvolvimento junto ao notebook Jupyter, somente devido ao ambiente gráfico junto ao navegador de internet, em termos corretos são chamados em lingua inglesa de "browser", mesmo sendo possível realizar todas as operações junto ao RStudio 2015.05. Necessitamos esclarecer que  todos as bibliotecas de funções no R são chamadas de "package", e ao carregarmos o pacote, " \textbf{\textcolor{blue}{raster}}" e "\textbf{\textcolor{blue}{terra}}", ao software pelo comando.
   \lstset{
 	language=R, % Define a linguagem como JavaScript
 	caption=Código para carregar imagens   em linguagem R,} % Legenda do código
\begin{lstlisting}[language=R]
library(raster)
library(terra)
raster_template <- terra::rast("D:\\IMAGENS2\\lc_20160202.tif")	   
\end{lstlisting} 
 \hspace*{1.25 cm} O package "RSToolsBOX", atualmente na versão, 1.0.2.1, traz uma série de ferramentas para trabalharmos com dados de sensoriamento remoto, tanto que este vem incorporado com as funções e/ou equações matemáticas nas 2 a 5. E dentro destas funções temos a função  "\textbf{\textcolor{blue}{spectralIndices}}", usada no script a seguir :
%
   \lstset{
	language=R, % Define a linguagem como JavaScript
	caption=Código para obter índices  em linguagem R,} % Legenda do código
\begin{lstlisting}[language=R]
VI_20131023 <- spectralIndices(raster_template, blue = "Layer_1", gree = "Layer_2" ,red = "Layer_3", nir = "Layer_4",swir2 =  "Layer_5", indices = c("NDVI", "NDWI","EVI2", "MSAVI") 
VI_palette <- brewer.pal(n = 10, name = "Spectral")
spplot(VI_20131023, col.regions = VI_palette, cuts = 6, col = "transparente")
\end{lstlisting} 
%
 \hspace*{1.25 cm} Este procedimento, visto que as duas imagens estão georreferenciadas ao mesmo sistema  de coordenadas, é observarmos por meios matemáticos a variação da reflectividade do mesmo ponto, por índices físicos medidos pelo sensor remoto, diminuindo a subjetividade do observador, e ao mesmo tempo qualificar e variância temporal da superfície.\\
 %
 %   
 \hspace*{1.25 cm} Tendo a imagens \ref{fig:ima2013} a  \ref{fig:inda2023}, estas em nível de cinza, e iniciarmos operações algébricas de obter diferenças entre duas imagens temporais em Figuras \ref{fig:rplot-ndwi2016} a \ref{fig:difer202332026}. \\
 %
 %
\hspace*{1.25 cm}  O próximo passo, ao inserir o "grid", malha regular espaçada a inicialmente a 1.600(mil e seiscentos) metros, para iniciar as operações de interpolação. Duas respostas desejamos obter nesta etapa,  uma superfície continua, e uma superfície que seja possível demonstrar direções do fenômeno estudado. E por fim, sua validade. \\
% 
\hspace*{1.25 cm} Por que de  forma complementar, a validação cruzada, permite  verificar o modelo gerado, esta aderente, ao observado e calculado. E por isso as técnicas de interpolação com métodos geoestatísticos se completam. 
\subsubsection{Geostatistica}
 % 
 \hspace*{1.25 cm} Inseridos os vértices do grid, pela biblioteca(package) \textbf{\textcolor{blue}{sp}}, quase todas as ferramentas para interpolação tem como passo fundamental o carregamento e perfeito funcionamento da biblioteca \textbf{\textcolor{blue}{stars}} e mais especificamente o \textbf{\textcolor{blue}{gstat}}, Posto que a interpolação ocorre entre a coordenadas geográfica e o valor mensurado. Além do mais, que este possibilita realizarmos a interpolação pelo inverso das distância, superfícies de tendência e krigagem.  O procedimento no R é: 
 \lstset{
	language=R, % Define a linguagem como JavaScript
	caption= Interpolação em linguagem R,} % Legenda do código
\begin{lstlisting}[language=R]
	library(stars)
	library(gstat)
	modelo_raster <- raster("00_datos\\SUBTRACAO.tif", values=FALSE)
	interpolado <- gstat::gstat(formula= variavelmensurada~1, locations = arquivodogrid, set = parametrosdomodelo )	   
\end{lstlisting}  
 % 
\hspace*{1.25 cm}  A terceira linha, é a imagem subtraída das alterações temporais, a ser demonstrada na Figura \ref{fig:difer202332026}, sendo necessária a todos os procedimentos inclusive a krigagem e superfícies de tendencia.\\
 % 
\hspace*{1.25 cm}  O resultado deste modelo, uma superfície continua, já nos possibilita termo o valor observado, o predito, seus resíduos, zscore, incluído suas coordenadas. Mais a possibilidade de monitorarmos a raiz do erro médio quadrado, e a estimativa da porcentagem da variação explicada $ R^{2}$.\\
%%
\hspace*{1.25 cm}  E neste momento, o processo torna-se interativo, testando os vários métodos de interpolação(simples interpolação, inverso das distância, inverso da distância com pesos ponderados, superfície de tendência e a krigagem). Enfatizando, por  termos os valores medido e estimado todos as medidas de variabilidade (media, mediana,moda, medidas de dispersão, distribuição de frequência ...)  \\
%
\hspace*{1.25 cm}  Profissionais que desenvolvem modelos matemáticos, muitas vezes apresenta a dificuldades em estabelecer as funções de transformação para normalizar seus dados. A inclusão  do package/biblioteca \textbf{\textcolor{blue}{bestnormalize}}, auxiliou demasiadamente em nosso modelo, realizando suas estimativas automaticamente. A sua implementação a liguagem R procedeu-se:
 \lstset{
	language=R, % Define a linguagem como R
	caption= normalizacao do modelo em linguagem R,} % Legenda do código
\begin{lstlisting}[language=R]
 library(bestNormalize)
modelo_normal <- bestNormalize(arquivodogrid$variavelmensurada )
\end{lstlisting}  
\hspace*{1.25 cm}  Segundo \cite[P.35]{Yamamoto} o termo variograma, ou semivariograma, ocorre um confusão terminológica de acordo com a literatura utilizada, para estes autores preferimos o exposto por \cite[p.228]{Ferreira}.  \begin{quoting}[rightmargin=0cm,leftmargin=2cm]
	\begin{singlespace}
		{
	\textit{Semivariogramas são modelos gráficos utilizados para se detectar o grau de dependência espacial entre dados geográficos em diferentes intervalos de distâncias crescentes, predefinidos e contados a partir de uma posição espacial inicial qualquer. Este modelo se constitui na curva da função de variância y(h) dos dados, onde h é um intervalo de distância até uma origem arbitrária(.. )}
		}
	\end{singlespace}
\end{quoting}
%
\hspace*{1.25 cm} A  parcela da modularização dos nossos procedimentos e analise do variograma para nossos dados foram obtidos pelo código a seguir.  
 \lstset{
	language=R, % Define a linguagem como R
	caption= Producao do variogramas em linguagem R,} % Legenda do código
\begin{lstlisting}[language=R]
	library(gstat)
   variograma <- gstat::variogram(variavelmensurada~1, arquivodogrid)
\end{lstlisting}  
\hspace*{1.25 cm}  E por fim a krigagem agora utilizando o package, sendo os valores da listagem em  \textbf{\textcolor{blue!55!black}{automap}} que também faz de forma automatizada. start\_vals somente para apresentar no figura desejada.
  \lstset{
 	language=R, % Define a linguagem como R
 	caption= Auto ajuste do Variograma em linguagem R,} % Legenda do código
 \begin{lstlisting}[language=R]
   library(automap)
 variogram_auto <- autofitVariogram(variavelmensurada~1, arquivodogrid, start_vals=c(variogramar$nugget, variogramar$cov.pars[2], variograma$cov.pars[1]))
 \end{lstlisting}  
 
  \lstset{
	language=R, % Define a linguagem como R
	caption= Mapa variografico em linguagem R,} % Legenda do código
\begin{lstlisting}[language=R]
	var_exp_map <- gstat::variogram(g, 	cutoff = 10000,	width = 1400, 	map = T)
\end{lstlisting}  
 
   \lstset{
 	language=R, % Define a linguagem como R
 	caption= Variograma utilizando funcao esférica em linguagem R,} % Legenda do código
 \begin{lstlisting}[language=R]
 	fit.sph <- gstat::fit.variogram(var_exp, vgm(contri1,"Sph", Range ,  0.01))
 \end{lstlisting}  
 
  \lstset{
	language=R, % Define a linguagem como R
	caption= Validação cruzada em linguagem R,} % Legenda do código
\begin{lstlisting}[language=R]
	xvalid.sph <- krige.cv(SUBTRACAO~1,locations = img1,model = fit.sph ) 
\end{lstlisting}  

  \lstset{
	language=R, % Define a linguagem como R
	caption= Krigagem em linguagem R,} % Legenda do código
\begin{lstlisting}[language=R]
	krigagem_auto <- autoKrige(SUBTRACAO~1, modelo_grid,  new_data=modelo_grid, start_vals=c(variofit_geor$nugget, variofit_geor$cov.pars[2], variofit_geor$cov.pars[1]))
\end{lstlisting}  
 
\subsubsection{Validação cruzada do modelo predito }
 
  \lstset{
	language=R, % Define a linguagem como R
	caption= Teste Lilliefors (Kolmogorov\_Smirnov)em linguagem R,} % Legenda do código
\begin{lstlisting}[language=R]
	lillie.test(valida3$pred)
\end{lstlisting}  
 
   \lstset{
 	language=R, % Define a linguagem como R
 	caption= Teste de correlação em linguagem R,} % Legenda do código
 \begin{lstlisting}[language=R]
 	cor.test(Mod_esfe$pred, Mod_esfe$observed, method = "pearson")
 	 \end{lstlisting} 
 
 
\hspace*{1.25 cm} Realizado a análise exploratória dos dados, estabelecido o modelo matemático que melhor se ajusta aos  fenômenos, podemos iniciar a inferência sobre nossos resultados. 
%   
   \begin{comment}
   \begin{wrapfigure}{r}{0.60\textwidth}
  	\begin{center}
  		\centering  \small \caption{Uso do solo em 2024}
  		\includegraphics[width=0.97\linewidth]{FIGURAS/Area-de-estudo}
  		\label{fig:area-de-estudo} \\{ Fonte:   Elaborado pelos Autores (2025)}
  	\end{center}
  \end{wrapfigure}
    \end{comment}
     \begin{comment}
\begin{figure}[H]
		\centering  \small \caption{Fluxograma de procedimentos}
		\includegraphics[width=0.47\linewidth]{DIAGRAMAS/flow-diagrama-Alterado2}
		%\caption{\href{file:./DIAGRAMAS/flow-diagrama-Alterado2.tex}{TEX File} }
		\label{fig:flow-diagrama-Alterado2}\\
		{ Fonte:   Elaborado pelos Autores (2025)}
\end{figure}  
        \end{comment}
% ------
%!TEX root = ..//DInamica-temporal-espacial.tex

\section{ Resultados }
 
 \subsection{Índices espectrais}
 
\hspace*{1.25 cm} O critério inicial de nosso procedimentos se estabeleceu inicialmente pela estabilidade dos indicadores bióticos do local. Tendo o  estado inicial em $ T_{0} =  2013$, sem ação antrópica do fenômeno, em $T _{1} =  2016$ o local já sofrendo a ação do fenômeno externo, e o conjunto da situação, mais próximo do momento atual em que o fenômeno se encerou $T _{2} =  2023$. \\
\hspace*{1.25 cm} Devido a situações como efeitos atmosféricos de  cobertura de nuvens, brumas. Impossibilitou a obtenção de imagens de resoluções temporais menores.\\ 
 % {{{   
 			\begin{minipage}[t!]{0.31\textwidth}
 				\begin{figure}[H]
 					\centering \small \caption{2013}
 					\includegraphics[width=0.97\linewidth]{FIGURAS/indices20131023}
 					\label{fig:ima2013} 
 				\end{figure}			
 				
 			\end{minipage}\hfill
 			\begin{minipage}[t!]{0.31\textwidth}
 				
 				\begin{figure}[H]
 					\centering \small \caption{2016}
 					\includegraphics[width=0.97\linewidth]{FIGURAS/indices20160101}
 					\label{fig:inda15} 
 				\end{figure}			
 				
 			\end{minipage} 
 			\begin{minipage}[t!]{0.31\textwidth}
 				
 				\begin{figure}[H]
 					\centering \small \caption{2023}
 					\includegraphics[width=0.97\linewidth]{FIGURAS/indices20231222}
 					\label{fig:inda2023} 
 				\end{figure}		
 			\end{minipage} 
 			
 			\begin{center}
 				Fonte:   Elaborado pelos Autores (2025)
 			\end{center}
 			% }}}
 \hspace*{1.25 cm} Neste conjunto de imagens em nível de cinza,  escolhido a paleta de cores "RdYlGn" do "package ViridisLite" nas Figuras \ref{fig:rplot-ndwi2016} a \ref{fig:difer202332026}, cujas as  medidas pelo sensor e transformadas aos índices espectrais, demonstram que a água ocorreu maior variação neste índice medido e calculado, e por nisso o índice NDWI  foi escolhido para uma análise mais priorizada.   \\
  % {{{   
 			\begin{minipage}[t!]{0.31\textwidth}
 				\begin{figure}[H]
 					\centering \small \caption{NDWI 2016}
 					\includegraphics[width=0.97\linewidth]{FIGURAS/Rplotndwi2016}
 					\label{fig:rplot-ndwi2016}
 				\end{figure}			
 				
 			\end{minipage}\hfill
 			\begin{minipage}[t!]{0.31\textwidth}
 				
 				\begin{figure}[H]
 					\centering \small \caption{NDWI 2023}
 					\includegraphics[width=0.97\linewidth]{FIGURAS/Rplot0ndwi2023}
 					\label{fig:rplot0ndwi2023}
 				\end{figure}			
 				
 			\end{minipage} 
 			\begin{minipage}[t!]{0.31\textwidth}
 				
 				\begin{figure}[H]
 					\centering \small \caption{Diferenças no NDWI}
 					\includegraphics[width=0.97\linewidth]{FIGURAS/difer202332026}
 					\label{fig:difer202332026}
 				\end{figure}		
 			\end{minipage} 
 			
 			\begin{center}
 				Fonte:   Elaborado pelos Autores (2025)
 			\end{center}
 			% }}}		
	%	
 \hspace*{1.25 cm}  Ao destacarmos o índice NDWI nas duas imagens com diferença temporal de 8(oito)anos, temos a Figura   \ref{fig:difer202332026} e nesta a atenção pode ser observada no canto superior direito, e em outras superfícies com maior presença de água, como brejos e açudes, a alteração contrastante destas diferenças.  \\
 %------------------
  \begin{wrapfigure}{r}{0.65\textwidth}
	\begin{center}
		\centering  \small \caption{Amostragem em grid}
		\includegraphics[width=0.97\linewidth]{FIGURAS/mdtamostras}
		\label{fig:mdtamostras}\\{ Fonte:   Elaborado pelos Autores (2025)}
	\end{center}
\end{wrapfigure} 
\subsection{Geoestátistica}
 %------------------
  \hspace*{1.25 cm}  Neste momento, a percepção do objeto observado ser uma superfície, quase continua(excluídos vazios radiométricos), torna-se inerente ao fenômeno de estudo. E por ser uma superfície, com valores máximos, mínimos, direção, desvio padrão e variância do valor do índice com coordenada (x,y). Deve ser tratada como tal, e ser analisadas  por técnicas da geoestátistica.   \\
  %------------------
  \hspace*{1.25 cm}  Ao decimo sétimo e oitavo paragrafo da seção 3, ao descrevermos a estrutura de amostragem, e ao parágrafo anterior  que  compreendemos a estrutura do fenômeno com uma \underline{variável regionalizada}, e por isso voltamos a  \cite[p.10]{Yamamoto} :
\begin{quoting}[rightmargin=0cm,leftmargin=4cm]
	\begin{singlespace}
		{
			\textit{Os métodos geoestatísticos fornecem um conjunto de técnicas necessárias para entender a aparente aleatoriedade dos dados, os quais apresentam, porém, uma possível estruturação espacial, estabelecendo, desse modo, uma função de correlação espacial.}
		}
	\end{singlespace}
\end{quoting} 
 %------------------
 \hspace*{1.25 cm} Ainda em  \cite[p.19 e 20]{Yamamoto}, o fenômeno estudado se comporta dentro de domínios, sejam em 2D ou 3D, e apresentam distribuição e variabilidade espaciais. A "metodologia da geoestátistica se destaca ao oferecer a incerteza associada à estimativa" \\
 %
  \hspace*{1.25 cm} Continuando em  \cite[p.20 e 21]{Yamamoto}, na "\textit{reprodução das característica do fenômeno espacial baseado em pontos amostrais é denominado interpolação ou estimativa}". O processo de para inferir a distribuição e a variabilidade espacial, vai  depender do tamanho da amostra e da distribuição espacial, e em nosso caso se estabeleceu tanto por aleatoriedade Figura \ref{fig:usoSOLOamostras} e Figura \ref{fig:mdtamostras} \\
 %
  \hspace*{1.25 cm}  Coletado os dados, e tabulados, necessitamos fazer a sumarização e analise exploratória,  no entanto varias vezes verificamos  situações que impossibilita termos conclusões sobre os seus resultados. E como em  \cite[p.18]{Webster}, muitas variáveis ambientais, principalmente nas  concentradas no solo, estão na forma de distribuição normal, de acordo com as "\textit{descoberta independentemente por De Moivre, Laplace e Gauss, mas Gauss parece geralmente levar o crédito por ela. E a distribuição é frequentemente chamada de "gaussiana"}". \\
%
  \hspace*{1.25 cm}  Ainda em \cite[p.20]{Webster}, mas a utilização destes dados na formulação de modelos, apresentam dificuldades, podemos transformar os valores medidos a uma nova escala, "\textit{se necessário, transformar os resultados para a escala original ao final}." \\
 % 
\hspace*{1.25 cm}  Na Figura \ref{fig:difer202332026} realizando operações de interpolação com a utilização do grid, conseguimos obter uma imagem com alguma normalidade, Figura \ref{fig:RplotN16}. No entanto ao aplicarmos o \textbf{\textcolor{blue}{bestnormalize}}, os resultado é surpreendente, tanto que apresentamos em Figura \ref{fig:RplotMELNHOR}, e nos estimulou a geração do modelo matemático em 3D, Figura \ref{fig:RplotP34}\\
%---------------- 
  % {{{   
 			\begin{minipage}[t!]{0.31\textwidth}
 				\begin{figure}[H]
 					\centering \small \caption{Antes da normalização}
 					\includegraphics[width=0.97\linewidth]{FIGURAS/RplotN1}
 					\label{fig:RplotN16}
 				\end{figure}			
 				
 			\end{minipage}\hfill
 			\begin{minipage}[t!]{0.31\textwidth}
 				
 				\begin{figure}[H]
 					\centering \small \caption{Apôs a normalização}
 					\includegraphics[width=0.97\linewidth]{FIGURAS/RplotMELNHOR}
 					\label{fig:RplotMELNHOR}
 				\end{figure}			
 				
 			\end{minipage} 
 			\begin{minipage}[t!]{0.31\textwidth}
 				
 				\begin{figure}[H]
 					\centering \small \caption{Perspectiva da Superfície}
 					\includegraphics[width=0.97\linewidth]{FIGURAS/RplotP34}
 					\label{fig:RplotP34}
 				\end{figure}		
 			\end{minipage} 
 			
 			\begin{center}
 				Fonte:   Elaborado pelos Autores (2025)
 			\end{center}
 			% }}}
 %%--------------
 \lstset{
	language=R, % Define a linguagem como JavaScript
	caption= Sujestões de melhoria ao modelo em  linguagem R} % Legenda do código
\begin{lstlisting}[language=R]
 Best Normalizing transformation with 64 Observations
Estimated Normality Statistics (Pearson P / df, lower => more normal):
- arcsinh(x): 1.7857
- Center+scale: 1.8024
- Double Reversed Log_b(x+a): 1.7667
- Exp(x): 1.9929
- Log_b(x+a): 1.5042
- orderNorm (ORQ): 1.4643
- sqrt(x + a): 1.2955
- Yeo-Johnson: 1.2452
Estimation method: Out-of-sample via CV with 10 folds and 5 repeats 
Based off these, bestNormalize chose:
Standardized Yeo-Johnson Transformation with 64 nonmissing obs.:
Estimated statistics:
- lambda = -3.643204 
- mean (before standardization) = -0.09376124 
- sd (before standardization) = 0.1402468
\end{lstlisting}  
% 
\hspace*{1.25 cm} A produção inicial do variograma,  com procedimentos em \cite[p.224]{Bivand}, com a caracterização mensurada e sua distância em visão inicial da análise em Figura \ref{fig:variogramageor4}\\
 %%%===================
 
 \begin{comment}
 	  \begin{minipage}[t!]{0.5\textwidth}
 	 	\begin{figure}[H]
 	 		\centering \small \caption{Semivariograma sem definição do modelo explicativo}
 	 		\includegraphics[width=0.97\linewidth]{FIGURAS/variogramaPTONOS}
 	 		\label{fig:variinical}
 	 	\end{figure}	
 	 \end{minipage}\hfill
 	 \begin{minipage}[t!]{0.5\textwidth}
 	 	\begin{figure}[H]
 	 		\centering \small \caption{Modularização da reflectância}
 	 		\includegraphics[width=0.97\linewidth]{FIGURAS/normali}
 	 		\label{fig:Rplothdg}
 	 	\end{figure}		
 	 \end{minipage} 
 	 % {{{   			
 	 			\begin{center}
 	 				Fonte:   Elaborado pelos Autores (2025)
 	 			\end{center}
 	 			% }}
 \end{comment}

 %==========================================		
%\hspace*{1.25 cm} E seu modelo de superfície plana, pode ser apreciado em Figura \ref{fig:superficie} critérios que podem também ser observados em Figura \ref{fig:superficie1} e sua melhora e Figura  \ref{fig:Rplothdg}\\
 		 %%-------------- 
  \hspace*{1.25 cm} A escolha da função matemática que produza, e nos possibilite a explicar o modelo, pode ser encontrada em  \cite[p.90]{delgado}, e a mesma deve ser inserido os parâmetros e efeitos, com o significado em língua inglesa que são o patamar ("sill"), pepita("nugget"), comprimento("range") e sua contribuição("contribution") .	\\	
 		 %%-------------- 
% {{{   
			\begin{minipage}[t!]{0.31\textwidth}
				\begin{figure}[H]
					\centering \small \caption{ Escolha da função }
					\includegraphics[width=0.97\linewidth]{FIGURAS/semivariancaIMAG}
					\label{fig:variogramageor4}
				\end{figure}			
				
			\end{minipage}\hfill
			\begin{minipage}[t!]{0.31\textwidth}
				
				\begin{figure}[H]
					\centering \small \caption{Escolha da função }
					\includegraphics[width=0.97\linewidth]{FIGURAS/variogramageor4}
					\label{fig:varexpfitsp}
				\end{figure}			
				
			\end{minipage} 
			\begin{minipage}[t!]{0.31\textwidth}
				
				\begin{figure}[H]
					\centering \small \caption{Modelo esférico}
					\includegraphics[width=0.97\linewidth]{FIGURAS/modelos-esferico}
					\label{fig:xvalid.cruzadaesferica2}
				\end{figure}		
			\end{minipage} 
			
			\begin{center}
				Fonte:   Elaborado pelos Autores (2025)
			\end{center}
			% }}
 \hspace*{1.25 cm} 	Em todas as Figuras \ref{fig:variogramageor4} e \ref{fig:superficie-de-tendencia} que a estacionáriedade  inicia-se visualmente em aproximadamente em 0,9 e seu "range" em torno de 1300 metros\\
 %		
 \hspace*{1.25 cm} Na Figura \ref{fig:variogramageor4} o processo da escolha da função, na Figura \ref{fig:varexpfitsp} a função escolhida, e finalmente na Figura \ref{fig:xvalid.cruzadaesferica2} a validação cruzada, com a regressão em homocedasticidade\\
 %%-------------- 
 \hspace*{1.25 cm} Em Figura \ref{fig:superficie-de-tendencia}, temos a superfície da interpolação, com a analise da covariância do modelo, distribuído espacialmente em Figura \ref{fig:superficie-de-tendencia} , com procedimentos vistos em   \cite[p.47]{delgado}, e como cada variável, seus resíduos, e observação contribuem na explicação do modelo em \ref{fig:Rplothddg} \\
   % {{{   
 			\begin{minipage}[t!]{0.31\textwidth}
 				\begin{figure}[H]
 					\centering \small \caption{Superfície interpolada}
 					\includegraphics[width=0.97\linewidth]{FIGURAS/superficie}
 					\label{fig:superficie-de-tendencia}
 				\end{figure}			
 				
 			\end{minipage}\hfill
 			\begin{minipage}[t!]{0.31\textwidth}
 				
 				\begin{figure}[H]
 					\centering \small \caption{Mapa variografico}
 					\includegraphics[width=0.97\linewidth]{FIGURAS/varexpmap}
 					\label{fig:mapavariografico}
 				\end{figure}			
 				
 			\end{minipage} 
 			\begin{minipage}[t!]{0.31\textwidth}
 				
 				\begin{figure}[H]
 					\centering \small \caption{Contribuição espacial das variável mensurada}
 					\includegraphics[width=0.97\linewidth]{FIGURAS/xvalid.sph-map}
 					\label{fig:Rplothddg}
 				\end{figure}		
 			\end{minipage} 
 			
 			\begin{center}
 				Fonte:   Elaborado pelos Autores (2025)
 			\end{center}
 			% }}

\hspace*{1.25 cm} Dando o enfoque no modelo esférico,  determinados enfase nos parâmetros deste modelo. E passo que escolhemos o modelo, sendo que o resultado erro médio quadrático  e a porcentagem explicada do modelo em quadra a seguir. 
%%
\lstset{
	language=R, % Define a linguagem como R
	caption= Resultado do modelo de superficie de tendencia de 3 grau saida da linguagem R} % Legenda do código
\begin{lstlisting}[language=R]
.0364868216565002  # root mean squared error
A matrix: 1 × 1 of type dbl
0.9465385  # % de explicacao do modelo
\end{lstlisting}   
 	\begin{figure}[H]
	\centering  \small \caption{Modelagem por krigagem}
	\includegraphics[width=0.4957\linewidth]{FIGURAS/MODELOKRIGAGEM-MELHOR}
	\label{fig:rplotkriga}\\{ Fonte:   Elaborado pelos Autores (2025)}
\end{figure}


 \begin{comment}
 	
 	
 	  \begin{wrapfigure}{l}{0.506\textwidth}
 		\begin{center}
 			\centering  \small \caption{Modelagem por krigagem}
 			\includegraphics[width=0.97\linewidth]{FIGURAS/MODELOKRIGAGEM-MELHOR}
 			\label{fig:rplotkriga}\\{ Fonte:   Elaborado pelos Autores (2025)}
 		\end{center}
 	\end{wrapfigure} 	 
 	

 	
 	%%%%%%%%%%%%%%%%%%%%%%%%%%%%%%%%%%%

 \end{comment}
  
\hspace*{1.25 cm} O que nos animou a elaboração da krigagem em Figura \ref{fig:rplotkriga}, o que possibilita ao lado direito da figura os locais onde ocorrem os maiores erros, na determinação desta superfície. E para também o variograma ficou próximo ao da Figura \ref{fig:Rplothddg} 
 
 \begin{comment}
\begin{figure}
	\centering
	\caption{}
	\label{fig:rplotndwi2016}
	\includegraphics[width=0.97\linewidth]{FIGURAS/Rplotndwi2016}
\end{figure}
 	\begin{center}
 		\includegraphics[width=0.97\linewidth]{FIGURAS/pontoB1}
 	\end{center}
 	\begin{center}
 		\includegraphics[width=0.97\linewidth]{FIGURAS/pontoC1}
 	\end{center}
 	\begin{center}
 		\includegraphics[width=0.97\linewidth]{FIGURAS/PontoD1}
 	\end{center}
 	
 \end{comment}
 
 \subsection{Analise da validação cruzada do Modelo }
 
   \lstset{
 	language=R, % Define a linguagem como R
 	caption= Resultado do Teste Lilliefors (Kolmogorov\_Smirnov) R,} % Legenda do código
 \begin{lstlisting}[language=R]
	Lilliefors (Kolmogorov-Smirnov) normality test
data:  valida3$pred
D = 0.024572, p-value = 0.5765 \end{lstlisting} 
 
   \lstset{
 	language=R, % Define a linguagem como R
 	caption= Resultado do Teste de Correlação em linguagem R,} % Legenda do código
 \begin{lstlisting}[language=R]	
 	Pearson s product-moment correlation
 	
 	data:  Mod_esfe$pred and Mod_esfe$observed
 	t = 24.657, df = 574, p-value < 2.2e-16
 	alternative hypothesis: true correlation is not equal to 0
 	95 percent confidence interval:
 	0.6750592 0.7546794
 	sample estimates:
 	cor 
 	0.7172019  \end{lstlisting}   
 
% {{{   
		 \begin{minipage}[t!]{0.32\textwidth}
				\begin{figure}[H]
					\centering  \small \caption{Distancia de Cooks e Alavancagem}
					\includegraphics[width=0.85\linewidth]{FIGURAS/dist-cook-alavan}
					\label{fig:DISTCOOK}
				\end{figure}
				
			\end{minipage}\hfill
			\begin{minipage}[t!]{0.332\textwidth}
					\begin{figure}[H]
				    \centering  \small \caption{Analise gráfica da  normalidade}
					\includegraphics[width=0.97\linewidth]{FIGURAS/valicas}
					\label{fig:valicas}
				\end{figure}
				
			\end{minipage}\hfill
			\begin{minipage}[t!]{0.32\textwidth}
					\begin{figure}[H]
					\centering  \small \caption{Analise gráfica da  correlação entre variáveis}
					\includegraphics[width=0.87\linewidth]{FIGURAS/painel-correla}
					\label{fig:painel-correla}
				\end{figure}
				
			\end{minipage} 
			 			\begin{center}
				Fonte:   Elaborado pelos Autores (2025)
			\end{center}
			% }}} 
 
 
 
\subsection{Dinâmica temporal }
\hspace*{1.25 cm} E finalmente, passamos para a análise temporal, o qual escolhemos uma posição geográfica onde ocorrem uma maio variação deste modelo superficial de variações de reflectância, pode análise temporal entre as diferenças  \( \Delta T = T_{2}  - T_{1} \)\\
%
\hspace*{1.25 cm}  Na Figura \ref{fig:INDICES}, mostra o comportamento anômalo de todos os índices espectrais ao inicio do ano de 2016, junto a margem.\\

 \begin{figure}[H]
	\centering  \small \caption{Índices em  análise temporal}
	\includegraphics[width=0.97\linewidth]{FIGURAS/indices}
	\label{fig:INDICES}{ Fonte:   Elaborado pelos Autores (2025)}
\end{figure}

 \begin{figure}[H]
	\centering  \small \caption{Aumento do óxido de ferro no período}
	\includegraphics[width=0.97\linewidth]{FIGURAS/graphvisualiser-1747928533564}
	\label{fig:INDICES2}{ Fonte:   Elaborado pelos Autores (2025)}
\end{figure}

\hspace*{1.25 cm} A variação do óxido de ferro alcançou o máximo ao ano de 2018 como Figura \ref{fig:INDICES2}\\
%
\begin{figure}[H]
	\centering  \small \caption{Índices NDVI e NDWI}
	\includegraphics[width=0.97\linewidth]{FIGURAS/graphvisualiser-1747928457345}
	\label{fig:INDICES3}{ Fonte:   Elaborado pelos Autores (2025)}
\end{figure}



% ------
%!TEX root = ..//DInamica-temporal-espacial.tex
 
\section{Discussão }

 \hspace*{1.25 cm} Na proposta de avaliação dos impactos do lançamento de rejeitos de mineração próximo a foz do rio Doce. Se o local escolhido do estudo, através da técnicas de sensoriamento remoto, nos possibilita inferir as alterações do meio físico: antes, durante e depois do acidente.\\
 %
 \hspace*{1.25 cm} Na primeira hipótese \textbf{\textcolor{blue}{$ H_{\ref{h1}}$}}, avaliou-se se o sensor utilizado consegue discriminar os efeitos do evento. Conforme apresentado nos parágrafos 1 a 4 da seção 4 e nas Figuras \ref{fig:rplot0ndwi2023} a \ref{fig:difer202332026},o índice NDWI revelou alterações perceptíveis na data próxima ao ocorrido, conforme ilustrado na Figura \ref{fig:inda2023}. \\
 %
 \hspace*{1.25 cm} Em segunda hipótese \textbf{\textcolor{blue}{$ H_{\ref{h2}}$}},, buscou-se uma explicação baseada em critérios matemáticos, com menor subjetividade na interpretação. Para tanto, utilizou-se a geoestatística como abordagem principal. \\
 %
 \hspace*{1.25 cm} Inicialmente, a amostragem aleatória de classes de uso do solo foi testada, mas os resultados, avaliados por meio de normalidade, erro amostral, variância, coeficiente de determinação e métrica RMSE, mostraram-se ineficientes, explicando apenas 41\% da variabilidade do modelo espacial.\\
 %
 \hspace*{1.25 cm} Por outro lado, a amostragem em grade ("grid") conseguiu explicar aproximadamente 95\% do modelo. Por isso, essa abordagem foi selecionada para a construção do semivariograma e sua representação espacial, incluindo os desvios-padrão.\\
  %
 \hspace*{1.25 cm} A amplitude geográfica e a maior variabilidade dos resíduos indicaram os locais onde o modelo de superfície tridimensional apresentou menor capacidade explicativa, sendo esses os pontos de maior atenção, com z-scores mais altos Figura \ref{fig:Rplothddg} e desvios mais expressivos, representados por regiões mais escuras no mapa da Figura \ref{fig:rplotkriga}.\\
 %
 \hspace*{1.25 cm} Portanto, a hipótesee \textbf{\textcolor{blue}{$ H_{\ref{h2}}$}} foi confirmada, indicando que o sensor foi sensível ao fenômeno e identificando áreas geográficas que demandam maior esforço de explicação.\\
 %
 \hspace*{1.25 cm} Por fim, a terceira hipótese \textbf{\textcolor{blue}{$ H_{\ref{h3}}$}}.foi considerada a mais direta pelos autores. A extração dos valores de reflectância e seus índices, diretamente sobre os alvos/focos em locais com maior variabilidade temporal, foi ocasionada pelo fenômeno e detectada pelo sensor. Em termos simples, o que foge à normalidade pode ser considerado anômalo.  \\
 %
\hspace*{1.25 cm} As coordenadas geográficas com menores valores de reflectância, em áreas sujeitas a alagamento, apresentaram as maiores distorções de reflectância ao longo do período analisado, conforme demonstrado nas Figuras \ref{fig:INDICES} a \ref{fig:INDICES3}, Após o evento, os valores de reflectância desses alvos retornaram à normalidade.
 
% ------
%!TEX root = ..//preambulo.tex
\section{Conclusão}

\hspace*{1.25 cm}  As Figuras \ref{fig:indao} ilustram a situação local durante a inundação ocorrida em 2013. Já as Figuras \ref{fig:indao2} e \ref{fig:indao3} mostram a chegada de rejeitos à foz do rio Doce. Ambos os fenômenos foram detectados por sensores orbitais e pela população residente.\\
%
\hspace*{1.25 cm} No presente estudo, a dinâmica territorial e espacial afetada pelo despejo de rejeitos de mineração foi analisada, consolidando a hipótese por meio de métodos matemáticos, especificamente a geoestatística, e de um painel temporal que descreve as alterações físicas do fenômeno..\\
% 
\hspace*{1.25 cm} Verificou-se que a amostragem por grade foi a mais apropriada para este estudo. Não ficou claro se a ausência de explicação sobre o modelo de amostragem aleatória decorreu de limitações metodológicas ou da falta de aprofundamento conceitual dessa técnica, uma vez que os procedimentos e algoritmos utilizados foram os mesmos.\\
%
\hspace*{1.25 cm}A função esférica ajustou-se melhor aos dados, e tanto o processo automático, e utilizando krigagem, quanto o processo manual obtiveram parâmetros semelhantes, permitindo descrever os locais geográficos para estudos mais aprofundados. \\
%
\hspace*{1.25 cm} A  Figura \ref{fig:Rplothdg} por meio da modularização para descrição e sumarização estatística da reflectância temporal, foi capaz de destacar pontos discrepantes relacionados ao fenômeno estudado. Esses pontos ou dados anômalos não foram excluídos na Figura \ref{fig:variinical}.\\
%
\hspace*{1.25 cm} Conforme recomendado pela própria metodologia geoestatística, estudos mais aprofundados devem ser direcionados aos locais onde o modelo apresentou dificuldades para explicar os fenômenos observados.

% ------
\addcontentsline{toc}{section}{Referências}
% ----------------------------------------------------------
% ELEMENTOS PÓS-TEXTUAIS
% ----------------------------------------------------------
%\postextual
%\newpage
% ----------------------------------------------------------
% Referências bibliográficas
% ----------------------------------------------------------
\bibliography{Referencias//abntex2-doc}
% ------


% ------
\PrintChanges
\PrintIndex
% ------

% ------
\StopEventually{\PrintIndex}
% ------

\StopEventually{\PrintIndex}
\end{document}

