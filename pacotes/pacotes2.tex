%!TEX root = preambulo.tex
% ---
% Pacotes de citações
% ---
%\usepackage[brazilian,hyperpageref]{backref}	 % Paginas com as citações na bibl
\usepackage[alf, % Permite ficar em ordem alfabetica
abnt-repeated-author-omit=true,  % Omitir autores com o mesmo nome
abnt-emphasize=bf, %Nome em destaque em negrito
abnt-etal-list=0, % utilizar o parentese em luga do colchete
]{abntex2cite}	% Citações padrão ABNT
\usepackage{nomencl} 			% Lista de simbolos
\usepackage{color}				% Controle das cores
\usepackage{nomencl} 			% Lista de simbolos
\usepackage{color}				% Controle das cores
\usepackage{graphicx,subcaption}		% Inclusão de gráficos
\usepackage{float} % Força o posicionamento da figura
\usepackage{microtype} 			% para melhorias de justificação
% ---
\usepackage{wrapfig}
\usepackage{blindtext} % Para gerar comentarios
\usepackage{tikz}
\usetikzlibrary{calc,positioning,arrows,shapes,shadows,fit,patterns,quotes,spy}
\usepackage{quoting}
\usepackage{setspace} % Este pacote altera o espaçamento entre linhas
\usepackage{amsmath} % modo matematico
% ---
\usepackage{todonotes} % Para comentarios magicos
\usepackage{comment}
\usepackage{verbatim}  % Comentários em multilinhas
\usepackage{listings}

\usepackage{xcolor} % Para cores personalizadas

\definecolor{keywords}{RGB}{255,0,90}
\definecolor{comments}{RGB}{0,0,113}
\definecolor{red}{RGB}{160,0,0}
\definecolor{green}{RGB}{0,150,0}
\definecolor{codegreen}{rgb}{0,0.6,0}
\definecolor{codegray}{rgb}{0.5,0.5,0.5}
\definecolor{codepurple}{rgb}{0.58,0,0.82}
\definecolor{backcolour}{rgb}{0.95,0.95,0.92}
\lstdefinestyle{mystyle}{
	backgroundcolor=\color{backcolour},   
	commentstyle=\color{codegreen},
	keywordstyle=\color{magenta},
	numberstyle=\tiny\color{codegray},
	stringstyle=\color{codepurple},
	basicstyle=\ttfamily\footnotesize,
	breakatwhitespace=false,         
	breaklines=true,                 
	captionpos=b,                    
	keepspaces=true,                 
	numbers=left,                    
	numbersep=5pt,                  
	showspaces=false,                
	showstringspaces=false,
	showtabs=false,                  
	tabsize=2
}
\lstset{style=mystyle}
\usepackage[framemethod=TikZ]{mdframed}
\mdfdefinestyle{MyFrame}{%
	linecolor=blue,
	outerlinewidth=2pt,
	roundcorner=20pt,
	innertopmargin=\baselineskip,
	innerbottommargin=\baselineskip,
	innerrightmargin=20pt,
	innerleftmargin=20pt,
	backgroundcolor=gray!40!white}


% ------------------------------------------------------