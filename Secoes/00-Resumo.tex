%!TEX root = ..//DInamica-temporal-espacial
\begin{center}
	\textbf{\textit{RESUMO}}
\end{center}
\noindent Análise dos efeitos locais, em perceptiva temporal  próximos a foz do rio Doce, pelo contato do desague de rejeitos oriundos do rompimento da barragem de Fundão MG. Pelos índices espectrais, mais precisamente o NDWI, determinou-se  a situação; antes, durante e depois, do lançamento dos efluentes. Por dois métodos de amostragem; aleatório e regular, avaliou-se por métodos geoestátisticos, posições onde ocorreu maior variabilidade do índice espectral estudado. Estabelecendo dois modelos matemáticos (superfície de tendencia de 3º e krigagem) para estudos mais aprofundados. Por fim, em painel temporal, demonstrou-se a variabilidade de vários índices(NDVI,IVEG,NDWI, EVI, MNDWI, oxido de ferro), a percepção destes alvo mensurado por sensor remoto Landsat 8. 

\noindent \textbf{\textcolor{blue}{Palavras chaves}  }:  \textit{ Espaço-temporal; índices espectrais  ;  superfícies de tendencia ; geoestátistica }

\begin{center}
\textbf{Spatiotemporal dynamics of contamination from the Fundão Dam collapse near the mouth of the Rio Doce}
\end{center}
\begin{center}
	\textbf{\textit{ABSTRACT}}
\end{center}
\noindent  
\noindent  Analysis of local effects, in temporal perception near the mouth of the Doce River, by the contact of the discharge of tailings from the rupture of the Fundão MG dam. By the spectral indices, more precisely the NDWI, the situation was determined; before, during and after the release of the effluents. By two sampling methods; random and regular, the positions where the greatest variability of the spectral index studied occurred were evaluated by geostatistical methods. Establishing two mathematical models (3º trend surface and kriging) for more in-depth studies. Finally, in a temporal panel, the variability of several indices (NDVI, IVEG, NDWI, EVI, MNDWI, iron oxide) was demonstrated, the perception of these targets measured by a Landsat 8 remote sensor.
\noindent\textbf{\textcolor{blue}{ Keywords}}:Spatiotemporal; spectral indices; trend surfaces; geostatistics 

\begin{comment}
	%
	\cite{Anderson}
	%
	\cite{Anderson}\\
	%
	\citep{Anderson}\\
	\citet{Anderson}\\
	%
	\citealp{Anderson}, \\
	\citealt{Anderson}, \\
	\citeauthor{Anderson}, \\
	\citeyear{Anderson}, \\
	\citeyearpar{Anderson}, \\
	\citeauthor*{Anderson}, \\
	\citep*{Anderson}, \\
	\citet*{Tanimoto}
\end{comment}
 