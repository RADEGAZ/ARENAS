%!TEX root = ..//DInamica-temporal-espacial.tex
\section{Conclusão}

\hspace*{1.25 cm}  As Figuras \ref{fig:indao} ilustram a situação local durante a inundação ocorrida em 2013. Já as Figuras \ref{fig:indao2} e \ref{fig:indao3} mostram a chegada de rejeitos à foz do rio Doce. Ambos os fenômenos foram detectados por sensores orbitais e pela população residente.\\
%
\hspace*{1.25 cm} No presente estudo, a dinâmica territorial e espacial afetada pelo despejo de rejeitos de mineração foi analisada, consolidando a hipótese por meio de métodos matemáticos, especificamente a geoestatística, e de um painel temporal que descreve as alterações físicas do fenômeno..\\
% 
\hspace*{1.25 cm} Verificou-se que a amostragem por grade foi a mais apropriada para este estudo. Não ficou claro se a ausência de explicação sobre o modelo de amostragem aleatória decorreu de limitações metodológicas ou da falta de aprofundamento conceitual dessa técnica, uma vez que os procedimentos e algoritmos utilizados foram os mesmos.\\
%
\hspace*{1.25 cm}A função esférica ajustou-se melhor aos dados, e tanto o processo automático, e utilizando krigagem, quanto o processo manual obtiveram parâmetros semelhantes, permitindo descrever os locais geográficos para estudos mais aprofundados. \\
%
\hspace*{1.25 cm} A  Figura \ref{fig:superficie-de-tendencia} por meio da modularização para descrição e sumarização estatística da reflectância temporal, foi capaz de destacar pontos discrepantes relacionados ao fenômeno estudado. Esses pontos ou dados anômalos não foram excluídos na Figura \ref{fig:RplotN16}.\\
%
\hspace*{1.25 cm} Conforme recomendado pela própria metodologia geoestatística, estudos mais aprofundados devem ser direcionados aos locais onde o modelo apresentou dificuldades para explicar os fenômenos observados.
