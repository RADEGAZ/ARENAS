%!TEX root = ..//preambulo.tex

\section{Introdução}

\hspace*{1.25 cm} O rompimento da barragem de Fundão, no povoado de Bento Rodrigues, município de Mariana, Minas Gerais, em 5 de novembro de 2015, lançou rejeitos de mineração de ferro na bacia hidrográfica do Rio Doce, causando uma série de prejuízos, incluindo perdas de vidas humanas e contaminação ambiental a jusante do corpo hídrico. \\
%
\hspace*{1.25 cm} O volume de rejeitos liberado, contendo material estranho ao bioma característico da região, alterou simultaneamente as condições dos corpos hídricos e da vegetação marginal..\\
%
\hspace*{1.25 cm}A cobertura desse evento foi amplamente noticiada pela imprensa nacional e internacional, incluindo veículos como \textit{The Wall Street Jornal}, \textit{The Guardian}, \textit{Le Monde}, entre outros.\\
%
\hspace*{1.25 cm} Em acordo com o Ministério publico (MP) e sua controladora a anglo-australiana  \textit{BHP Billinton} ,  baseado nos princípios de poluidor pagador, trazidos no artigo 3º, inciso IV, da Lei 6.938/81 (Politica Nacional do Meio Ambiente), estabeleceram um termo de pactuação e posterior outra repactuação.  A repactuação atendendo o principio do poluidor-pagador, visa o dever de corrigir, recuperar e/ou eliminar os efeitos negativos ja produzidos em uma contaminação antrópica. \\
% 
\hspace*{1.25 cm}  A repactuação, utilizando ferramentas de cartografia, acordo com \cite[p~79]{Magri}, determinando ações de reparação aos atingidos e a  amplitude de localização geográfica em que ocorram ações de contaminação, e  por meio de direta interveniência  judicial e administrativa, e a fundação criada para reparação do rio Doce, denominada Fundação Renova.\\
 %
\hspace*{1.25 cm} Passados nove anos, com base na data de 23 de maio de 2025, e com a redução da atenção midiática sobre o desastre, este estudo propõe como objeto de pesquisa avaliar se os efeitos da contaminação por rejeitos ainda podem ser detectados por sensores orbitais e se há alterações passíveis de observação por esses instrumentos. Esses aspectos são objetos de perícia ambiental, conforme descrito por   \cite[p.130]{Arantes} \\
 %
\hspace*{1.25 cm}  A justificativa deste estudo, motivado pela delimitação estabelecida pela repactuação entre o Ministério Público e a Fundação Renova. Esta foi a resposta à calamidade, as partes envolvidas – a Samarco, suas acionistas Vale e BHP Billiton, a União e os governos de Minas Gerais e do Espírito Santo – firmaram o Termo de Transação e Ajustamento de Conduta (TTAC) em 2 de março de 2016. Com base neste acordo, foi criada a Fundação Renova em 2016, uma entidade privada com a missão de conduzir as 42 ações e programas socioeconômicos e socioambientais definidos para a reparação. No entanto, ao longo dos anos, a atuação da Fundação Renova foi alvo de diversas críticas e gerou um passivo significativo de 85 mil processos judiciais, evidenciando a necessidade de uma solução mais eficaz e abrangente.
%Além disso, considera-se a análise dos enquadramentos dos corpos d’água e as diretrizes ambientais previstas na Resolução Conama nº 357, de março de 2005, alterada pelas Resoluções nº 410/2009 e nº 430/2011, que podem ser observadas em locais dentro e fora do âmbito da repactuação.\\
%

%\hspace*{1.25 cm} A tragédia do rompimento da barragem de Fundão, em Mariana/MG, ocorrida em 5 de novembro de 2015, representa um dos maiores desastres socioambientais do Brasil. Este evento catastrófico lançou aproximadamente 39 milhões de metros cúbicos de rejeitos de mineração na Bacia do Rio Doce, causando a perda de 19 vidas e impactando severamente populações em dezenas de municípios até a foz no Espírito Santo. A amplitude do desastre e seus prejuízos humanos e ambientais ganharam notoriedade na imprensa nacional e internacional, como The Wall Street Journal, The Guardian e Le Monde.\\
%
%\hspace*{1.25 cm} Em 
%
%\hspace*{1.25 cm} Diante da complexidade e da insatisfação com o progresso reparatório, iniciaram-se em março de 2021 as tratativas para uma renegociação ampla dos acordos, formalizadas pela Carta de Premissas em 22 de junho de 2021. Este processo, conduzido inicialmente pelo Conselho Nacional de Justiça (CNJ) e, a partir de agosto de 2022, sob a liderança do Tribunal Regional Federal da 6ª Região (TRF6) através da Mesa de Repactuação, buscou encerrar os múltiplos litígios por meio de um procedimento de conciliação. Após quase três anos de intensas negociações, o Acordo Judicial para Reparação Integral e Definitiva Relativa ao Rompimento da Barragem de Fundão foi finalmente assinado em Brasília em 25 de outubro de 2024 e homologado por unanimidade pelo Supremo Tribunal Federal (STF) em 6 de novembro de 2024.\\
%
%\hspace*{1.25 cm} Este novo acordo, que substitui integralmente o TTAC de 2016 e seus aditivos, busca a reparação integral e definitiva de todos os danos socioambientais e socioeconômicos. Um dos seus pilares é a extinção da Fundação Renova e do Comitê Interfederativo (CIF), transferindo a responsabilidade integral pelas ações de reparação diretamente para a Samarco, que iniciará um período de liquidação para essa transição. O acordo prevê um valor econômico total de R\$ 170 bilhões, que inclui:
\begin{description} [itemsep=1pt,parsep=1pt]\vspace{0.00mm} 
	\item[•] R\$ 38 bilhões já desembolsados desde a tragédia.
	\item[•] R\$ 100 bilhões em "dinheiro novo", destinados aos entes públicos para custeio de medidas compensatórias e projetos socioambientais e socioeconômicos.
	\item[•] R\$ 32 bilhões em "Obrigações de Fazer", que incluem indenizações individuais, reconstrução de comunidades e recuperação de áreas degradadas, sem um teto financeiro pré-determinado, devendo a Samarco comprovar a conclusão de cada obrigação.
\end{description}
%
%\hspace*{1.25 cm} A nova governança do processo reparatório será marcada pela transparência, com a criação de um "Portal Único" denominado "Reparação Rio Doce", onde todas as partes envolvidas (signatários) serão responsáveis pela atualização dos dados, permitindo à sociedade civil acompanhar detalhadamente a implementação do acordo. A homologação do acordo também resultará na extinção de inúmeras ações judiciais e procedimentos administrativos, visando a definitiva resolução dos litígios.
\hspace*{1.25 cm} Com base na exposições acima, as seguintes hipóteses são formuladas:
\begin{description} [itemsep=1pt,parsep=1pt]\vspace{0.00mm} 
	\item[$H_{1}$:\label{h1}] O sensor orbita Landsat 8 consegue identificar alterações em respostas em radiância na época  do ocorrido, e seus efeitos ainda podem ser mensurados
	\item[$H_{2}$:\label{h2}] Se este sensor orbital consegue distinguir alterações espectral  do fenômeno, e o mesmo consegue determinar sua amplitude espacial geográfica, de forma a caracteriza-la.
	\item[$H_{3}$:\label{h3}] Existe resquícios desta contaminação no local, e que podem ser mensurados.
\end{description}
