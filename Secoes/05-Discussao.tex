%!TEX root = ..//preambulo.tex
 
\section{Discussão }

 \hspace*{1.25 cm} Na proposta de avaliação dos impactos do lançamento de rejeitos de mineração próximo a foz do rio Doce. Se o local escolhido do estudo, através da técnicas de sensoriamento remoto, nos possibilita inferir as alterações do meio físico: antes, durante e depois do acidente.\\
 %
 \hspace*{1.25 cm} Na primeira hipótese \textbf{\textcolor{blue}{$ H_{\ref{h1}}$}}, avaliou-se se o sensor utilizado consegue discriminar os efeitos do evento. Conforme apresentado nos parágrafos 1 a 4 da seção 4 e nas Figuras \ref{fig:rplot0ndwi2023} a \ref{fig:difer202332026},o índice NDWI revelou alterações perceptíveis na data próxima ao ocorrido, conforme ilustrado na Figura \ref{fig:inda2023}. \\
 %
 \hspace*{1.25 cm} Em segunda hipótese \textbf{\textcolor{blue}{$ H_{\ref{h2}}$}},, buscou-se uma explicação baseada em critérios matemáticos, com menor subjetividade na interpretação. Para tanto, utilizou-se a geoestatística como abordagem principal. \\
 %
 \hspace*{1.25 cm} Inicialmente, a amostragem aleatória de classes de uso do solo foi testada, mas os resultados, avaliados por meio de normalidade, erro amostral, variância, coeficiente de determinação e métrica RMSE, mostraram-se ineficientes, explicando apenas 41\% da variabilidade do modelo espacial.\\
 %
 \hspace*{1.25 cm} Por outro lado, a amostragem em grade ("grid") conseguiu explicar aproximadamente 95\% do modelo. Por isso, essa abordagem foi selecionada para a construção do semivariograma e sua representação espacial, incluindo os desvios-padrão.\\
  %
 \hspace*{1.25 cm} A amplitude geográfica e a maior variabilidade dos resíduos indicaram os locais onde o modelo de superfície tridimensional apresentou menor capacidade explicativa, sendo esses os pontos de maior atenção, com z-scores mais altos \ref{fig:Rplothddg} e desvios mais expressivos, representados por regiões mais escuras no mapa da Figura \ref{fig:rplotkriga}.\\
 %
 \hspace*{1.25 cm} Portanto, a hipótesee \textbf{\textcolor{blue}{$ H_{\ref{h2}}$}} foi confirmada, indicando que o sensor foi sensível ao fenômeno e identificando áreas geográficas que demandam maior esforço de explicação.\\
 %
 \hspace*{1.25 cm} Por fim, a terceira hipótese \textbf{\textcolor{blue}{$ H_{\ref{h3}}$}}.foi considerada a mais direta pelos autores. A extração dos valores de reflectância e seus índices, diretamente sobre os alvos/focos em locais com maior variabilidade temporal, foi ocasionada pelo fenômeno e detectada pelo sensor. Em termos simples, o que foge à normalidade pode ser considerado anômalo.  \\
 %
\hspace*{1.25 cm} As coordenadas geográficas com menores valores de reflectância, em áreas sujeitas a alagamento, apresentaram as maiores distorções de reflectância ao longo do período analisado, conforme demonstrado nas Figuras \ref{fig:INDICES} a \ref{fig:INDICES3}, Após o evento, os valores de reflectância desses alvos retornaram à normalidade.
 