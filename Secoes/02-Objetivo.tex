%!TEX root =..//preambulo.tex

\section{ Objetivos}
%
\subsection{ Objetivos Gerais}
%
\hspace*{1.25 cm} O objetivo deste estudo é comparar medidas de reflectância no solo, em vegetações e na água, utilizando a tecnologia de sensoriamento remoto (SR), para diferenciar as condições geográficas antes, durante e após o evento em análise. Além disso, busca-se verificar a possibilidade de identificar locais com contaminações e/ou alterações antrópicas decorrentes do desastre de Mariana
%\hspace*{1.25 cm} Com , verificar se é possível discriminar locais com contaminações resultantes do desastre de mariana. 
 % 
\subsection{ Objetivos Específicos}

 \hspace*{1.25 cm} A estruturação das informações será dividida em etapas. Inicialmente, serão obtidos dados por meio de imagens orbitais, armazenados, processados e analisados.Para isso, será utilizada a plataforma de acesso livre da "\textit{Google Earth Enginer(GEE)}". E como descreve em \cite[p.1]{Mutanga}, motivados por essa plataforma permitir o acesso a séries de imagens Landsat a partir de 2008.\\
%
 \hspace*{1.25 cm} O segundo passo consiste na realização de uma amostragem sistemática em locais geográficos específicos, utilizando valores de reflectância obtidos pela plataforma GEE. Essa abordagem possibilitará a construção de um painel temporal da ocorrência do evento.\\
% 
 \hspace*{1.25 cm} Por fim, a modelagem do fenômeno será realizada com base em critérios de validação estatística, que fornecerão subsídios para uma amostragem mais refinada, utilizando métodos probabilísticos aplicados a classes de uso do solo em zonas específicas\\
% 
% \hspace*{1.25 cm} Por fim, a obtenção de amostras in-loco de informações sobre padrões de indicadores químicos
 
 % \hspace*{1.25 cm} \textcolor{gray!22}{\lipsum[1-2] }\cite{IJSN}
  

	
